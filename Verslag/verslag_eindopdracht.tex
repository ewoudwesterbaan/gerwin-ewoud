\documentclass[a4paper]{article}

\usepackage{xcolor}
\usepackage{fancyheadings}


\newcommand{\todo}[1]{\textcolor{red}{[#1]}}
\lhead{Open Universiteit}
\chead{IM0102, Design patterns}
\rhead{Eindopdracht}

\begin{document}
\pagestyle{fancy}

\section*{Studentgegevens}
\begin{description}
	\item [Cursuscode] IM0102
	\item Scenario 6: Meerdere slidesets
	\item [Naam] \todo{invullen}
	\item [Studentnummer] \todo{invullen}
	\item [Naam] \todo{invullen}
	\item [Studentnummer] \todo{invullen}
\end{description}

\section*{Aanpak}
\todo{<Geef aan hoe jullie de opdracht hebben aangepakt en wie wat heeft gedaan, maximaal 1 A-4. Geef expliciet aandacht aan de volgorde van activiteiten>}
Uit leereenheid 3 de huidige beschrijving gehaald
Deze uitgebreid met de concepten uit de scenario beschrijving



\section{Probleemanalyse}
\subsection{Dingen}
\begin{description}
\item[Presentatie] is de weergave van gespecificeerde slides in een vaste volgorde. Slides kunnen meerdere keren voorkomen. Een presentatie bevat een titel. Deze is altijd zichtbaar tijdens de weergave van de presentatie. In het programma kunnen meerdere presentaties beschikbaar zijn.
\item[Slide] is de weergave van een pagina op het scherm. Een slide heeft een titel, een nummer, slide items en heeft een bepaalde oppervlakte.
\item[Slide item] is een element op een slide. Slide items worden onder elkaar getoond en is van een bepaalde type.
Een slide item wordt in een stijl te gepresenteerd.
\item[Stijl] is een vorm van weergave van een slide item in termen van horizontale uitlijning, lettertype, kleur, en lettertypegrootte.
\end{description}

\subsection{Regels}
\begin{description}
\item[Presentatie] bepaalt de volgorde van de slides
(Een presentatie bepaalt wat van de slides weergegeven moet worden.)
\item[Slide item] bepaalt de stijl van de weergave.
\item[Stijl] bepaalt doe een slide item wordt weergegeven. 
\end{description}

\subsection{Verantwoordelijkheden}
\begin{description}
\item[Presentatie] is verantwoordelijk voor de correcte volgorde van de slides en het beschikbaar stellen hiervan. 
\item[Slide] kan zijn titel tonen en de slide items schalen en onder elkaar laten tekenen.
\item[Slide item] kan zichzelf tekenen.
\item[Inlezen] Slides verzameling opbouwen en presentaties opbouwen (delegeren).
\item[View] Het laten zien van een slide aan de gebruiker.
\item[Applicatie] Inlezen van slide verzameling en presentaties hiervan.
\end{description}

\subsection{Aannames}
Bestaande presentaties moeten ook ingelezen kunnen worden. Indien er geen presentaties zijn, kunnen alle slides weergegeven worden.

\section{Ontwerp}
\begin{description}
\item[Presenter] is verantwoordelijk voor de omgang van meerdere presentaties. Het biedt afnemers de mogelijkheid om een presentatie te kiezen.
\item[Presentation] is verantwoordelijk voor de correcte volgorde van de slides, het beschikbaar stellen hiervan en afnemers laten weten dat de slide beschikbaar is. Het biedt mogelijkheden om door slides te navigeren.
\item[Slide] is verantwoordelijk voor het tekenen van zichzelf.
\item[SlideItem] Zorg dragen voor het tekenen van items.
\item[Item] is verantwoordelijk voor het tekenen van een TextItemof BitmapItem.
\item[PresenterObserver] verantwoordelijk voor het reageren op wijziging van de presentatie. Bijvoorbeeld het veranderen van een presentatie.
\item[PresentationObserver] verantwoordelijk voor reageren op het wijzigen van de aangeboden slide (dat er een andere slide ter beschikking wordt gesteld).
\item[Loader] Laden van presentaties vanaf een persistente opslag.
\end{description}

\section{Keuzen}
\begin{description}
\item[Model-View-Controller] als architectuur ontwerp. Het model is verantwoordelijk voor het kenbaar maken van wijzigingen. Dit wordt gedaan door het Observer patroon twee keer toe te passen.
\item[Observer] Voor het kenbaar maken dat een andere slide getoond moet worden, hebben we gekozen voor het observer pattern. De reden is dat in de toekomst mogelijk andere observers ook acties moeten uitvoeren op het moment dat een andere slide getoond wordt. Te denken is aan een presentatorscherm waar opmerkingen op getoond worden. Tevens hebben we gekozen voor dit patroon bij het kenbaar maken dat de presentation gewijzigd is. In dit geval is het de Presenter die PresenterObservers inlichten.
\begin{itemize}
\item Subject: Presentation / Presenter
\item Observer: PresentationObserver / PresenterObserver
\end{itemize}
\item[Iterator] Voor het aanbieden van slides aan de afnemer hebben we gekozen voor een iterator patroon. De afnemers, alle PresentationObservers moet beschikking hebben over de objecten (Slides).
\begin{itemize}
\item Aggregator: Presenter
\item Iterator: Presentation
\item ConcreteAggregator: PresentationManager
\item ConcreteIterator: SlideSequence en SlideDeck
\end{itemize}
We hebben dit patroon niet toegepast bij de slideitems, omdat de afnemer niet direct toegang hoeft te hebben tot de slideitems.

\item[Strategy] De SlideItem heeft een item en een stijl. De slideitem geeft de stijl aan de item om deze zich te laten tekenen. De implementatie van het tekenen met een gegeven stijl kan anders zijn voor de verschillende type items.
\begin{itemize}
\item Strategy: Item
\item ConcreteStrategies: TextItem, BitmapItem
\item Context: SlideItem
\end{itemize}

\item[Stijl] \todo{Decorator?}
\end{description}
\subsection{Toekomstige veranderingen}
\begin{description}
\item[Bepaalde delen laten zien] Het laten zien van bepaalde delen van een slide is gegeven als een mogelijke verandering. Dit is te implementeren door een nieuwe concrete klasse te definiëren die de Slide interface implementeerd. Doordat deze klasse een associatie heeft naar de SlideDeckSlide kan de klasse de SlideItems benaderen en in de implementatie van de draw() methode hoeven dan alleen deze specifieke SlideItems opgehaald te worden.
\item[Meerdere views] Zoals al aangegeven bij de observer pattern, is het mogelijk om bijvoorbeeld een aparte presentator view te maken waar opmerkingen op te zien zijn.
\item[Andere type items] Het is mogelijk om andere type items te tonen op een slide, door een nieuwe concrete klasse te maken die de Item interface implementeerd.
\end{description}


\section{Sourcecode}

\end{document}
