\documentclass[a4paper]{article}

\usepackage{xcolor}
\usepackage{fancyheadings}


\newcommand{\todo}[1]{\textcolor{red}{[#1]}}
\lhead{Open Universiteit}
\chead{IM0102, Design patterns}
\rhead{Eindopdracht}

\begin{document}
\pagestyle{fancy}

\section*{Studentgegevens}
\begin{description}
	\item [Cursuscode] IM0102
	\item \todo{Titel van het scenario}
	\item [Naam] \todo{invullen}
	\item [Studentnummer] \todo{invullen}
	\item [Naam] \todo{invullen}
	\item [Studentnummer] \todo{invullen}
\end{description}

\section*{Aanpak}
\todo{<Geef aan hoe jullie de opdracht hebben aangepakt en wie wat heeft gedaan, maximaal 1 A-4. Geef expliciet aandacht aan de volgorde van activiteiten>}
Uit leereenheid 3 de huidige beschrijving gehaald
Deze uitgebreid met de concepten uit de scenario beschrijving



\section{Probleemanalyse}
\subsection{Dingen}
\begin{description}
\item[Slide verzameling] is een verzameling van slides (een ‘bak’ van slides).
\item[Presentatie] is de weergave van slides in een bepaalde volgorde. De presentatie haalt de slides die nodig zijn uit de slides verzameling.
\item[Slide] is de weergave van een pagina op het scherm.
Een slide heeft een titel, een nummer en slideitems.
\item[Slide item] is een element op een slide. Slideitems worden onder elkaar getoond en is van een bepaalde type.
Een slideItem heeft een nesting level. Deze wordt gebruikt om het item in een stijl te presenteren. (Een slideitem heeft een nummer.)
\item[Stijl] is een vorm van weergave van een slideitem in termen van horizontale uitlijning, lettertype, kleur, en lettertypegrootte.
\end{description}

\subsection{Regels}
\begin{description}
\item[Presentatie] bepaalt de volgorde van de slides
(Een presentatie bepaalt wat van de slides weergegeven moet worden.)
\item[Slide item] bepaalt de stijl van de weergave.
\item[Stijl] bepaalt doe een slideitem wordt weergegeven. 
\end{description}

\subsection{Verantwoordelijkheden}
\begin{description}
\item[Slide verzameling] is verantwoordelijk voor het beheren van de slides en het teruggeven van een slide.
\item[Presentatie] is verantwoordelijk voor de volgorde van de slides en deze aanbieden.
\item[Inlezen] Slides verzameling opbouwen en presentaties opbouwen (delegeren).
\item[View] Het laten zien van een slide aan de gebruiker.
\item[Applicatie] Inlezen van slide verzameling en presentaties hiervan.
\end{description}


\section{Ontwerp}


\section{Keuzen}
\begin{description}
\item[Graphische gebruikers interface] Model-View-Controller?
\item[Itereren door slides] Iterator?
\item[SlideItem] Strategy?
\item[Slide verzameling] PoolFactory?
\item[Stijl] Decorator?
\item[PresentationSlide] Adapter?
\end{description}

\section{Sourcecode}

\end{document}
